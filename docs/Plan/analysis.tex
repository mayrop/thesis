%////////////////////////////////////////////////////////////////////////
%   C O N F I G
%////////////////////////////////////////////////////////////////////////

\documentclass[
12pt, 
english, 
oneside,
doublespacing, % Single line spacing, alternatives: onehalfspacing or doublespacing
%draft, % Uncomment to enable draft mode (no pictures, no links, overfull hboxes indicated)
nolistspacing, % If the document is onehalfspacing or doublespacing, uncomment this to set spacing in lists to single
%liststotoc, % Uncomment to add the list of figures/tables/etc to the table of contents
%toctotoc, % Uncomment to add the main table of contents to the table of contents
%parskip, % Uncomment to add space between paragraphs
%nohyperref, % Uncomment to not load the hyperref package
headsepline, % Uncomment to get a line under the header
%chapterinoneline, % Uncomment to place the chapter title next to the number on one line
%consistentlayout, % Uncomment to change the layout of the declaration, abstract and acknowledgements pages to match the default layout
]{../Thesis} % The class file specifying the document structure

% Margins

\geometry{
	paper=a4paper,
	inner=2.5cm,
	outer=3.8cm,
	bindingoffset=.5cm,
	top=1.5cm,
	bottom=1.5cm,
	head=24pt
	%showframe, % Uncomment to show how the type block is set on the page
}

% \setlength{\headsep}{2cm}

\usepackage[utf8]{inputenc} % Required for inputting international characters
\usepackage[T1]{fontenc} % Output font encoding for international characters

% Bibliography
\usepackage[backend=biber,style=numeric,sorting=ynt]{biblatex}

% Required to generate language-dependent quotes in the bibliography
\usepackage{fvextra}
\usepackage[autostyle=true]{csquotes}
\usepackage[utf8]{inputenc}

\addbibresource{thesis.bib}

\begin{document}

 % Pre-content pages, roman numbers
\frontmatter

% Default plain style 
\pagestyle{plain}

%////////////////////////////////////////////////////////////////////////
%   C H A P T E R S
%////////////////////////////////////////////////////////////////////////

% Begin numeric (1,2,3...) page numbering
\mainmatter

{\bfseries\noindent\huge{Statistical Analysis Plan}}

{\noindent\Large{Assessing the impact of socio-economic factors on Presidential Primary Election voting in the USA in 2016}} % Main chapter title

%----------------------------------------------------------------------------------------

\vspace{3ex}
{\bfseries\noindent\large{Population}}

\noindent{U.S.A. Population.}

%----------------------------------------------------------------------------------------

\vspace{3ex}
{\bfseries\noindent\large{Primary Objective}}

- Find if there are specific socio-economic or demographic factors that are associates with an increased or decreased preference for a political party by county.

- Find associations between specific socio-economic or demographic factors and the fraction of people voting for a republican candidate by county.

%----------------------------------------------------------------------------------------

\vspace{3ex}
{\bfseries\noindent\large{Secondary Objectives}}

- Find state-wide factors that are associated with a preference for one political party over another.

- Prediction of the final state-wide outcome of the presidential election in 2016. For this an outcome by county will be needed and then grouped by state.

%----------------------------------------------------------------------------------------

\vspace{3ex}
{\bfseries\noindent\large{Data Collection}}

- Demographic data on counties from U.S.A. Census Bureau

- MIT Election Data and Science Lab, 2018, "County Presidential Election Returns 2000-2016", https://doi.org/10.7910/DVN/VOQCHQ, Harvard Dataverse, V2


%----------------------------------------------------------------------------------------

\vspace{3ex}
{\bfseries\noindent\large{Variables Under Consideration}}

{\noindent\bfseries{Outcome variables}}

\noindent{Could be either the data on winning party (binary variable) or the proportion of votes won (in \%). Note if the proportion of votes is used the proportion of total votes emitted may be needed as well.}

{\noindent\bfseries{Covariates}}

\noindent{There are in total fifty explanatory variables, where: 
- Eighteen of them are demographic variables relating to the population and racial  composition

- Two of them relate to educational attainment

- One of them correspond to the number of war veterans

- Seven relate to housing; fourteen relate to income and employment; five of them to 
sales

- Three of them to building permits, land area and population per square mile (density)

\noindent{All of the explanatory variables are continous and most of them relate to the percentage of the population while some of them are raw counts.}
%----------------------------------------------------------------------------------------

\vspace{3ex}
{\bfseries\noindent\large{Missing Data Procedures}}

\noindent{Unlikely to be an issue. One of the states of the U.S.A. (Alaska) has different ways of getting census data and elections votes, so analysis may exclude Alaska.}

%----------------------------------------------------------------------------------------

\vspace{3ex}
{\bfseries\noindent\large{Summaries to be presented}}

- Scatterplots showing the preference of the association for variables and the increase / decrease preference for a particular party.

- Maps (univariate/bivariate) showing the relationships of the data.

- Descriptive statistics for the variables of interest.

%----------------------------------------------------------------------------------------

\vspace{3ex}
{\bfseries\noindent\large{Models to be fitted}}

- Regression models may be used for the find associations for an increase or decrease preference for a particular party. However, feature selections will need to be done carefully using maybe PCA or hierarchical clustering.

- After investigation, it seems that SVM (Support Vector Machines) may be suitable for classification for the prediction objective.

%----------------------------------------------------------------------------------------

\vspace{3ex}
{\bfseries\noindent\large{Risks}}

- Dataset has the percentage of republican data. One of the assumptions is that there are only two parties in the election. However, this percentage would need to be re-escaled from republican / democrat if we exclude the percentage of the "other" parties. Other option is to take all the parties in mind.




%////////////////////////////////////////////////////////////////////////

\end{document}  
